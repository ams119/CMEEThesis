\section{Introduction}
\setcounter{page}{1}
%\pagenumbering{arabic}

%\subsubsection{Introduce}

At the end of the 20th century, the world experienced a global resurgence of vector-borne disease \citep{Gubler}. Diseases that had been well-controlled in the early-to-mid 1900s surged under complacent public health policies and insufficient research funding \citep{Gubler1998}. %Many of these diseases also emerged in novel geographic areas \citep{Gratz1999}. 
Today, vector-borne diseases represent 17\% of the total global infectious disease burden and cause millions of deaths annually. Mosquito-borne diseases, such as malaria, dengue, Zika, yellow fever, West Nile virus, and chikungunya, singularly infect more than an estimated 200 million individuals worldwide every year \citep{WHOreport}. 

Surveillance of mosquito populations is a successful method to control the public health impacts of vector-borne disease \citep{Vazquez-Prokopec2010}. Intervention in growing populations through chemical control measures can effectively reduce disease incidence \citep{Tomerini2011}. Sampling methods are, however, often limited by resource constraints \citep{Sedda2019}. Many studies have attempted to develop early warning models of disease incidence through prediction, rather than surveillance, of mosquito abundance \citep{Beck-Johnson2013, Li2019, Poh2019}. Because localised arbovirus disease data is often lacking in quality, mosquito abundance models can offer an alternative method to estimate disease risk \citep{Lowe2013}. Predictive models of mosquito abundance could also offer a cost-effective strategy with which to plan control measures \citep{Yang2009}.  
%In order to understand the potential applications of early warning mosquito abundance models, greater investigation is needed into optimal methodologies of vector abundance models across locations and species. 
%. 

Mosquito abundances are affected by many factors, including land-use, elevation, and vegetation cover, but meteorological variables, such as temperature and precipitation, are particularly predictive of population dynamics and commonly used in abundance models \citep{Trawinski2008, Li2019, Wang2011, Yoo2016}. The effects of temperature on mosquitoes are well characterised. As ectotherms, mosquito life history traits such as development rate, biting rate, fecundity, and survival, are temperature dependent and vary with changing temperatures \citep{Mordecai2019}. With increasing temperature, trait performance will increase until an optimal temperature is reached, after which, trait performance decreases according to physiological constraints \citep{Amarasekare2012}. Because these traits shape reproductive output, trait variation determines abundance of mosquito populations \citep{Cator2020}. Precipitation affects mosquitoes in more complex ways. Rainfall can create larval habitats in man-made containers or expand natural pooled breeding habitats \citep{Keith2005, Koenraadt2008}. Inter-seasonal variability in larval carrying capacity, which is dependent on rainfall for habitat creation, has been shown to be a main driver of mosquito abundances \citep{Marini2016}. Heavy rainfall, however, can flush immature mosquitoes from aquatic habitats, but the extent of this effect varies among species \citep{Koenraadt2008, Paaijmans2007}. The effect of droughts on abundance is mixed; in some cases, drought is thought to benefit mosquito abundance by eliminating predators of larvae in drying water bodies \citep{Chase2003}. Other research have found that droughts simply increase sample collection, rather than true abundance of mosquitoes \citep{Shaman2002}. In light of projected increases in extreme weather events such as increased heavy rainfall frequency, droughts, and warming temperatures in this century due to climate change, it is vital that we understand the complex effects of temperature and precipitation on vector populations. \citep{Seneviratne2012}.

A wide variety mechanistic and statistical modelling techniques have been used to investigate the impact of climate on mosquito abundance \citep{Ahumada2004, Cailly2012, Jian2014, Yoo2016, Wang2011}. Mechanistic models based on trait responses to climate conditions may allow for longer term and more generalisable forecasting \citep{Cator2020}. The complexity of the relationship between precipitation and mosquito life history, however, complicates the inclusion of this important abundance driver in mechanistic models. Statistical models, on the other hand, may be more useful for understanding  explanatory power of environmental variables \citep{Mordecai2019}. Mosquito abundances have been described phenomenologically by harmonic models, time series ARIMA models, Generalised Linear Models (GLMs), and Generalised Additive Models (GAMs) \citep{Li2019, Trawinski2008, Wang2011, Yoo2016}. GAMs have recently been used to capture the complex and non-linear relationships between climate and abundance as linear models are likely to overestimate the effect of temperature on mosquito abundance at high temperatures \citep{Li2019, Roiz2014, Xu2017}. 

Several unanswered methodological questions about the use of statistical models could limit the applicability of these methods as early warning systems of mosquito abundance and vector borne disease. Temperature and precipitation have both been found to have temporally lagged impacts on mosquito abundance at daily, weekly, and monthly scales \citep{Chuang2012, Poh2019, Xu2017, Wang2011}. In other words, past precipitation and temperature can drive contemporary mosquito population dynamics. Although lags of different time scales capture different effects of environmental drivers, tested methodologies for appropriate temporal resolutions to use for lagged models are non-existent \citep{Mordecai2019}. Most statistical models have also focused on characterising lags and dependence on meteorological variables in single species and limited geographic areas \citep{Chuang2012, Wang2011, Poh2019, Yoo2016}. Because abundance responses to meteorological drivers are also dependent on local land characteristics such as elevation, vegetation cover, and urban structures, climate effects are likely to be location-specific \citep{Ahumada2004, Yoo2016}. In order to understand the applicability of early warning systems for vector species abundances across geographies, the consistency of temporal lag lengths and meteorological abundance drivers in abundance models needs to be assessed across several locations. The inclusion of a first-order auto-regressive term (AR) of lagged abundance in analysis of GAMs can improve model fits by capturing density dependence of mosquito population dynamics and auto-correlative structure of mosquito abundances. \citep{DaCruzFerreira2017, Li2019, Xu2017}. The need for nearly contemporary abundance data for this AR term, however, could limit the real-time forecasting ability of these methods. Research is needed into the trade off in model fit with the use of AR terms to understand how applicable these models are in real time. This study aims to address these gaps in statistical modelling of mosquito abundance dynamics by answering the following questions:

\begin{enumerate}
	\item Which level of temporal resolution of temperature and precipitation data is best able to predict mosquito abundances?
	
	\item How consistently can temporal lags and significant meteorological drivers be characterised for species specific abundance models across multiple locations?
	
	\item How does incorporation of an auto-regressive term affect the predictive ability of temperature and precipitation driven models of mosquito abundance?
\end{enumerate}

To answer these questions, I will be evaluating temperature and precipitation dependent multivariate GAMs of mosquito abundance constructed using surveillance datasets from five geographically dispersed locations in Florida. This study is novel for its use of comprehensive multi-species and multi-location datasets. The insights gained from this investigation should help to inform the development of cost-effective early warning systems of increases of mosquito abundances and vector-borne-disease. 

%In this investigation I will take advantage of an open source mosquito abundance database to access comprehensive multi-species surveillance records in Florida, U.S.A. Florida's plentiful surveillance records reflect public health concerns stemming from a long history of mosquito-borne disease epidemics, including yellow fever, dengue, and malaria \citep{Connelly2014}. Since the start of the 21st century, this state has seen a re-emergence of dengue fever, geographically novel emergence of West Nile Virus, and locally transmitted outbreaks and incidences of Eastern Equine Encephalitis, St. Louis Encephalitis, Zika, and Chikungunya fever \citep{Connelly2014},\textbf{\cite{FloridaHealth2017}}. I first mapped openly-sourced datasets of precipitation and temperature data to Florida surveillance data and aggregated these datasets to weekly, bimonthly, and monthly time scales. I then fit univariate GAMs of the effect of each meteorological variable on mosquito abundance independently at multiple temporal lags. Model selection was used to find the best fit temporal lag length for each dataset, meteorological variable, and temporal resolution. Best fit lags of temperature and precipitation were then incorporated into a multivariate GAM of mosquito abundance. The fit and predictive ability of these multivariate models were compared across temporal resolutions. Consistency of  selected lag lengths and significance of meteorological drivers was then compared within 11 species across five Florida locations. Finally, I incorporated an AR term into multivariate GAMs and compared the relative fit and performance of auto-regressive and non-autoregressive models across datasets.



%More sustainable systems are needed for vector-borne disease surveillance \citep{Vazquez-Prokopec2010}

%At the end of the 20th century, the world experienced a rising global threat of vector-borne diseases (VBD) \citep{Gubler}. The incidence of many VBDs, such as malaria, yellow fever, and dengue, that had been locally eradicated in the early-to-mid 1900s, resurged under complacent public health policies and a lack of research funding \citep{Gubler1998} Of the many diseases threatening human health, mosquito borne diseases such as malaria, yellow fever, dengue, West Nile Virus, and Rift Valley fever take millions of lives every year \citep{Yang2009}. Mosquito abundances are affected by many factors, including land-use, elevation, and vegetation cover, but meteorological variables such as temperature and precipitation in particular can be predictors of population dynamics \citep{Yoo2016}. Rainfall produces basins of water for breeding while temperature mediates life-history processes at all life stages \citep{Yang2009, Beck-Johnson2013}. 