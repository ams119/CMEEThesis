\section{Results}

Figures/Charts: 

1. time series of abundance, temp, and precipitation
\begin{figure}[p]

		\begin{minipage}[]{.51\textwidth}
			\includegraphics[height = 3in, width=3.2in]{../Images/temp_ts.pdf}
			
			\hspace{4.3cm}\textbf{A.}\\
		\end{minipage}
		\begin{minipage}[]{.47\textwidth}
			\includegraphics[height = 3in, width=3.2in]{../Images/precip_ts.pdf}
			
			\hspace{4.4cm}\textbf{B.}\\
		\end{minipage}
	
	
		\begin{minipage}[]{.5\textwidth}
			\includegraphics[height = 3in, width=3.2in]{../Images/aedes_abun_ts.pdf}
			
			\hspace{4.3cm}\textbf{C.}\\
		\end{minipage}
		\begin{minipage}[]{.5\textwidth}
			\includegraphics[height = 3in, 			width=3.2in]{../Images/culex_abun_ts.pdf}
			
			\hspace{4.5cm}\textbf{D.}\\
		\end{minipage}
		
	\label{fig: ts_plots}
	\caption{Sample time series data of temperature, precipitation, and two vector species from Manatee County, Florida, 2012-2012. Data shown here are aggregated at the weekly scale. In \textbf{A.}, average daily maximum temperature per week shows regular seasonality with a range of about 15 \degree C. In \textbf{B.}, precipitation is very frequent during during a summer rainy season. This is common in all locations.\textbf{C.} and \textbf{D.} show patterns of abundance for two vector species, \textit{Aedes albopictus} and \textit{Culex nigripalpus}. Abundance datasets were non-continuous over winter periods where abundances are assumed to be low so traps are not employed.}

\end{figure}

2. partial dependency plot of multivariate GAM of a sample species 

\begin{figure}
	\begin{minipage}[]{0.5\textwidth}
		\includegraphics[height = 3.5in, width=3.5in]{../Images/multi_plotAQ.png}
		
		\hspace{4.5cm}\textbf{A.}\\
	\end{minipage}
	\begin{minipage}[]{0.5\textwidth}
	\includegraphics[height = 3.5in, width=3.5in]{../Images/multi_plotAA.png}
	
	\hspace{4.5cm}\textbf{B.}\\
	\end{minipage}

	\label{fig: 3Dmulti}
	\caption{This a sample of datasets fit with  multivariate models of temperature and precipitation at the best fit lags (Equation \ref{eq: multivariate1}). In \textbf{A.}, average \textit{Anopheles quadrimaculatus} count per trap at a weekly resolution was best fit according to univariate model comparison (Equations \ref{eq: univariateT} and \ref{eq: univariateP}) with temperature at a lag of six weeks and precipitation with a lag of four weeks. Both temperature and precipitation were significant. In \textbf{B.}, a Aedes albopictus abundance at a monthly resolution was best fit with temperature at a lag of one month  and precipitation in the contemporary month. In this dataset, only precipitation was significant.}

\end{figure}


\begin{figure}
	\centering
	\includegraphics[height = 3.5in, width=7in]{../Images/multi_plot2.png}
	\caption{This is my second option for demonstrating multivariate model fits (versus Fig. \ref{fig: 3Dmulti})}
\end{figure}

3. Table comparing deviance explained of different temporal resolutions

\begin{table}[h]
\begin{center}
\begin{tabularx}{.8\textwidth}{| L{0.5}  C{0.5}  C{.5}  C{0.5} | }
	\hline
	\multicolumn{4}{|l|}{Median Deviance Explained at Various Temporal Resolutions}\\
	\hline
	County & Weekly & Biweekly & Monthly \\
	\hline
	Lee & 31.9\% & 42.2\% & *52.2\% \\
	Manatee & 24.8\% & 21.6\% & *34.1\% \\
	Orange & 17.0\% & 15.8\% & *21.3\% \\
	Saint Johns & *8.5\% & 7.4\% & 8.4\% \\
	Walton & 23.3\% & 40.0\% & *49.9\% \\
	\hline
	All Counties & 20.0\% & 22.2\% & *31.9\% \\
	\hline
	
\end{tabularx}
\label{tab: bestres}
\caption{This table makes me hate Saint Johns}

\end{center}
\end{table}


5. Plot showing effect of autoregressive term. Also include in words what the AIC difference was between AR and non-AR model

\begin{figure}
	\centering
	\includegraphics[width= 5in, height = 3in]{../Images/devcomp.pdf}
\end{figure}