\documentclass[11pt, a4paper]{report}
\usepackage[margin = 2cm]{geometry}
\usepackage[utf8]{inputenc}
\usepackage{graphicx} %import graphics
% import pdf :\includegraphics[<options>]{filename.pdf}
\graphicspath{../../Results}
\usepackage{setspace}
\onehalfspacing % sets spacing to 1.5
\usepackage{titlesec}
\usepackage{caption}
\usepackage{subcaption}
\usepackage[export]{adjustbox} % adjust figure position
\usepackage{fixltx2e} % subscripts in text
\usepackage{natbib} %check out citation styles: https://www.economics.utoronto.ca/osborne/latex/BIBTEX.HTM
\setcounter{secnumdepth}{0} %removes numbering from all sections and levels below
\setlength{\parindent}{4em}
\usepackage{setspace}
\doublespacing
\usepackage{comment}
\usepackage[left]{lineno}
\usepackage{array}
\usepackage{gensymb} % For degree symbol
\usepackage{ltxtable}
\usepackage[table]{xcolor} 
\usepackage{amsmath, amssymb} % for expectation
\arrayrulecolor{gray}
\newcolumntype{L}[1]{>{\hsize=#1\hsize\raggedright\arraybackslash}X}
\newcolumntype{C}[1]{>{\hsize=#1\hsize\centering\arraybackslash}X}
\newcolumntype{R}[1]{>{\hsize=#1\hsize\raggedleft\arraybackslash}X}

% Change Figure to Fig.
\renewcommand{\figurename}{Fig.}

% Create expectation symbol
\DeclareMathOperator{\EX}{\mathbb{E}}

%Caption format
\captionsetup{
	width=\textwidth, % width of caption is 90% of current textwidth
	labelfont={bf, sc},        % the label, e.g. figure 12, is bold
	font=normalsize,          
	%format=hang,         % no caption text under the label
	justification=raggedright,
	singlelinecheck=false,
	labelsep=period
}


\titleformat{\section}{\normalfont\bfseries\LARGE}{}{0pt}{}
%\titleformat{\section}
%{\bfseries\Large} %formatting
%{} %numbering
%{0em} % distance between numbering and subtitle
%{} % anything to appear between # and title

\titleformat{\subsection}{\normalfont\bfseries\normalsize}{}{0pt}{}

\title{Evaluating Statistical Modelling Methodologies of Disease Vector Abundances } 

\author{Anne Marie Saunders\\
Imperial College London\\
MSc Computational Methods in Ecology and Evolution\\}

\date{August 27th, 2020\\ 
%Word Count: \input{sum.txt} words
}




\begin{document}
%\setcounter{page}{1}
%\pagenumbering{roman}

\begin{titlepage}
	\maketitle
\end{titlepage}


%\addcontentsline{toc}{section}{Declaration}
\renewcommand{\abstractname}{Declaration}
\begin{abstract}
Mosquito count data is from the VectDyn database of the VectorByte data platform. Meteorological data is from the NOAA Climate Data Online database. Meteorological data extraction from NetCDF files was adapted from protocols developed by Matthew Watts as part of the larger VectorByte project. All other data cleaning, processing, and analysis was my own work.
\end{abstract}

%\addcontentsline{toc}{section}{Abstract}
\renewcommand{\abstractname}{Abstract}
\begin{abstract}
Mosquito population dynamics directly affect disease incidence. Climate-based statistical models of abundance dynamics could provide early warnings of mosquito population growth and inform public health and mosquito management strategy. Existing attempts to model the relationship of temporally lagged meteorological variables on mosquito abundances tend to focus on single species in limited geographic areas and vary in their incorporation of auto régressive terms in the model.Conclusions about appropriate temporal  lag lengths and significance of meteorological predictors are consequently highly specific and limited in their potential to inform wider reaching early warning models. The temporal resolution of these models also range from weekly to monthly time scales with little consensus on appropriate time scale of analysis. In this study I used comprehensive multi-species surveillance data from 5 Florida counties to compare the performance of temperature and precipitation based GAMs of abundance when aggregated at weekly, bimonthly, and monthly time scales. I also investigated the consistency of best fit temporal lag lengths and significance of temperature and precipitation in predicting abundance across multiple locations for the same species. Lastly I assessed the relative effect on model fit of incorporating an auto-regressive term into climate-based GAMs. I found that in most locations, models parameterised with monthly aggregated data explain the most deviance in mosquito abundance dynamics with the least absolute error. Lags and significance of meteorological drivers of mosquito abundance were commonly shared within species, but varied enough to recommend that statistical models should be location-specific. According to AIC selection, autoregressive models were, in a majority of cases, better fit than non-autoregressive models, but a substantial 31.4\% of datasets were equally or better fit by climate-only models.
\end{abstract}

\clearpage\tableofcontents %creates page without page number
\thispagestyle{empty}

\pagebreak %also try \newpage

%\begin{linenumbers}
\section{Introduction}
\setcounter{page}{1}
%\pagenumbering{arabic}

\subsubsection{Introduce}

At the end of the 20th century, the world experienced a global resurgence of vector-borne disease \citep{Gubler}. Diseases that had been well-controlled in the early-to-mid 1900s surged under complacent public health policies and insufficient research funding \citep{Gubler1998}. %Many of these diseases also emerged in novel geographic areas \citep{Gratz1999}. 
Today, vector-borne diseases represent 17\% of the total global infectious disease burden and cause millions of deaths annually. Mosquito-borne diseases, such as malaria, dengue, Zika, yellow fever, West Nile virus, and chikungunya, singularly infect more than an estimated 200 million individuals worldwide every year \citep{WHOreport}. 

Surveillance of mosquito populations is a successful and cost-effective method to control the public health impacts of vector-borne disease \citep{Vazquez-Prokopec2010}. Many studies have attempted to develop early warning models of disease incidence through prediction, rather than surveillance, of mosquito abundance \citep{Beck-Johnson2013, Li2019, Poh2019}. Intervention in the growth of mosquito popualtions through chemical control measures is an effective public health strategy for reducing incidence of mosquito borne disease \citep{Tomerini2011}. Timing these intervention measures to interrupt peak mosquito abundances, whether predicted or surveilled, could inform cost-effective disease management solutions 
%\citep{Yang2009}. 
Mosquito abundances are affected by many factors, including land-use, elevation, and vegetation cover, but meteorological variables, such as temperature and precipitation, are particularly predictive of population dynamics \citep{Yoo2016} 

Precipitation could have mixed effects on mosquito abundance dynamics. Precipitation raises near-surface humidity, which increases adult mosquito activity and host-seeking behaviour \citep{Shaman2007}. This increased reproductive activity would lead to lagged effects on mosquito abundance, but would also cause immediate increases in mosquito trap counts. Many trap types are designed to attract host-seeking or gravid females and are consequently also attractive to mate-seeking male mosquitoes \citep{Li2016}. Rainfall also affects many of the aquatic habitats important for early life stages \citep{Shaman2007}. Container-breeding species that use man-made containers for oviposition may experience increased breeding sites when precipitation creates habitats in otherwise dry containers \citep{Keith2005}. Precipitation can also expand suitable habitats for mosquitoes breeding in natural water bodies \citep{Koenraadt2008}. While some rainfall seems likely to have positive effects on mosquito abundance, the effect of heavy rainfall is less clear. Heavy rainfall can flush immature mosquitoes from aquatic habitats, but the extent of this effect varies among species \citep{Koenraadt2008, Paaijmans2007}. For example, specialist container-breeding \textit{Aedes aegypti} has been found to have a stronger protective diving response to rainfall compared to generalist habitat breeding \textit{Culex pipiens} \citep{Koenraadt2008}. %\textbf{Add sentence about changing (increasing) precipitation with climate change}.
	
Cover:
- temperature effects
- lags
- GAMS
- non-linear relationship between climate and mosquito abundance \citep{Roiz2014}

\subsubsection{Gaps}
- many species/locations
- use of autoregressive term or not
%Previous research has found t-1 autoregressive terms to be significantly associated with mosquito abundance \cite{Poh2019}(1 month), \citep{Xu2017} (1 month), 

- different temporal resolutions (so far restricted)
- lag determination across species/location


\subsubsection{Novelty of Research}

Justify choice of Florida

\begin{figure}[h]
	\centering
	\includegraphics[width= 6in, height = 4in]{../Images/arboviralcases.pdf}
	\caption{Number of locally transmitted cases of mosquito-borne disease in humans in Florida between 2014 and 2019, as reported by the Florida Department of Health in  \citeyear{DiseaseSurveillance}. Zika was also locally transmitted in 2016 (300 confirmed cases) and 2017 (2 confirmed cases) but are not shown due to the limited scale of the y axis. Imported cases, where infection was contracted elsewhere but reported in Florida, are not included as these were not contracted from Florida mosquitoes. %Imported cases are a threat to public health, however, as they can continue the disease transmission cycle and are a major concern for dengue and Zika.
	}
	\label{fig:FDcases}
\end{figure}

More sustainable systems are needed for vector-borne disease surveillance \citep{Vazquez-Prokopec2010}

Many species, many locations

assessing practical application by comparing predictive power of autoregressive vs non autoregressive model

\subsubsection{Clear-cut Questions}
\begin{enumerate}
	%\item Is mean precipitation or number of days of precipitation a better predictor of mosquito abundances?
	
	%\item Which temporal lags of temperature and precipitation are most appropriate for the prediction of mosquito abundances?

	\item \textbf{Which level of temporal resolution of temperature and precipitation data is best able to predict mosquito abundances?}
	
	\item \textbf{To what extent can species-specific abundance models be predictive at multiple locations?}
	
	\item \textbf{How does incorporation of an auto-regressive term affect the predictive ability of temperature and precipitation driven models of mosquito abundance?}
\end{enumerate}

%At the end of the 20th century, the world experienced a rising global threat of vector-borne diseases (VBD) \citep{Gubler}. The incidence of many VBDs, such as malaria, yellow fever, and dengue, that had been locally eradicated in the early-to-mid 1900s, resurged under complacent public health policies and a lack of research funding \citep{Gubler1998} Of the many diseases threatening human health, mosquito borne diseases such as malaria, yellow fever, dengue, West Nile Virus, and Rift Valley fever take millions of lives every year \citep{Yang2009}. Mosquito abundances are affected by many factors, including land-use, elevation, and vegetation cover, but meteorological variables such as temperature and precipitation in particular can be predictors of population dynamics \citep{Yoo2016}. Rainfall produces basins of water for breeding while temperature mediates life-history processes at all life stages \citep{Yang2009, Beck-Johnson2013}. 
\pagebreak

\section{Methods}

\subsection{Mosquito Abundance Data}

Mosquito count data was obtained from the VectDyn database (\textbf{HOW TO CITE?}). VectDyn is a global database containing spatially and temporally explicit abundance data of mosquitoes and other arthropod vectors. Mosquito abundance from 205 global locations was narrowed down to 7 data-rich counties in Florida, U.S.A. From these 7 counties, 5 counties were determined to have nearly year-round sampling from which fairly continuous time series of mosquito abundances could be formed. 

\begin{center}
	\begin{tabularx}{4in}{ | L{1.4} | C{0.6} | }
		\hline
		\textbf{County} & \textbf{Years of Data} \\\hline
		Lee & 11 \\\hline
		Manatee & 5 \\\hline
		Orange & 6 \\\hline
		Saint Johns & 13.5 \\\hline
		Walton & 3 \\\hline 
		\multicolumn{1}{|r|}{\textbf{Average}} & \textbf{7.7} \\\hline
	\end{tabularx}\\
	\captionof{table}{\label{Tab:Locs} Locations of abundance data and associated length of data location at each location. Individual species in these locations may have abundance records that are shorter than the overall collection duration of each location. }
\end{center}	

Each location contained multiple trap sites. A variety of trap types were used, including BG-Sentinel traps, CDC light traps, animal-baited traps, CDC gravid traps, and \textbf{collection of arthropods}. Because I was only concerned with relative changes in abundance and not absolute abundance across species or locations, I included all trap types in my data. Abundance was recorded in integer count values and usually identified to the species level. 


\subsection{Meteorological Data}

Temperature and precipitation datasets were obtained from the NOAA Climate Data Online database as NetCDF raster files at a spatial resolution of 0.50 degrees latitude and 0.50 degrees longitude. Maximum temperature in Celsius and total daily precipitation in millimetres were used based on availability of data. Rasters were rotated 180 degrees to match coordinate rotation of trap locations. Maximum daily temperature and total precipitation values were then extracted by taking the mean of the bilinear interpolation of the 4 closest raster cells to each trap location. I then mapped extracted maximum temperature and total precipitation values to corresponding mosquito abundances by date and trap location. 

\textbf{Number of days of rainfall has been shown to be a more effective predictor of mosquito-borne disease incidence than cumulative precipitation \citep{Xu2017}. This is likely due to the maintenance of humid conditions over time with frequent rainfall. Humidity has been independently assessed as a significant predictor of abundance dynamics \citep{Trawinski2008}, and so this representation of precipitation may capture both humidity and precipitation effects.}

\subsection{Data Pre-Processing}

In order to account for discrepancies between true zero count instances and NA values in abundance data, I set abundance for each species to zero where at least one mosquito of any species had been caught at the same trap on the same day. 

I then spatially and temporally aggregated meteorological data and species-level abundance data. At the spatial level, I averaged the maximum temperature, total precipitation, and species-specific abundance from trap-specific to county-wide. This transformed integer count values to averaged indicators of overall abundance for the county-level spatial scale. I aggregated morphological groups of non-differentiable species that could be easily mis-identified (\textbf{list in SI}). I then removed species that had only zero or NA abundance values. Species counts that were identified only to the genus or family level were also removed. I then temporally aggregated maximum temperature, total precipitation, and species-specific abundance by averaging at weekly, biweekly, and monthly scales. Consequently, temperature refers to the average maximum daily temperature across respective temporal scales and precipitation refers to average daily precipitation across the respective temporal scales. Finally, I removed rows of data with missing values in response or explanatory variables. 

\textbf{include paragraph on preprocessing for GAM: dealing with NAs, interpolation for autoregressive model}

\subsection{Model Structure}

I used univariate generalized additive models for each species at each location to determine the best-fit temporal lags between abundance and maximum temperature as well as between abundance and precipitation. Because aggregated abundance values are positive, non-integer, and non-normally distributed, I used a Gamma family distribution with a log-link function. These models had the form:
\begin{equation}
	ln(V_t + 1) = a_0 + f_1(T_{t-l_T}) + \epsilon_t
	\label{eq: univariateT}
\end{equation}
\begin{equation}
	ln(V_t + 1) = a_0 + f_1(P_{t-l_P}) + \epsilon_t
	\label{eq: univariateP}
\end{equation}


$V_t$ is abundance at time $t$, $a_0$ is the intercept, $T_{t-l_T}$ and $P_{t-l_P}$ are the temperature and precipitation at $l_T$ and $l_P$, respectively, time periods prior to time $t$. One is added to abundance at time $t$ in order to prevent undefined values from the logarithm of zero abundance values. $f_1$ is a smooth function comprised of cubic polynomial basis functions. I set the number of basis functions for each smooth function to four throughout this analysis in order to control overfitting of the GAM. 

Each dataset was also fit with two multivariate models, each incorporating temperature and precipitation, and one of which incorporating an autoregressive term. 

\begin{equation}
	ln(V_t + 1) = a_0 + f_1(T_{t-l_T}) + f_2(P_{t-l_P}) + \epsilon_t
	\label{eq: multivariate1}
\end{equation}
\begin{equation}
	ln(V_t + 1) = a_0 + f_1(T_{t-l_T}) + f_2(P_{t-l_P}) + f_3(V_{t-1} + 1) + \epsilon_t
	\label{eq: multivariate2}
\end{equation}

In Equation \ref{eq: multivariate2}, $V_{t-1}$ is the abundance one time step prior to abundance at time $t$. 

\subsection{Model Selection and Evaluation}

Models \ref{eq: univariateT} and \ref{eq: univariateP} were used to determine the best-fit lags of temperature and precipitation for each dataset at each temporal aggregation. At the weekly and biweekly scale, lags between zero and six weeks were considered. At the monthly scale, lags between zero and two months were considered. Best-fit lags for each temporal scale were chosen by finding the minimum Akaike's Information Criterion (AIC) value among models fit with each possible lag length. 

\begin{equation}
	AIC = -2ln[L(\hat{\theta_{p}}|y)] + 2p
	\label{eq: AIC}
\end{equation}

AIC is a method for determining the likelihood of a given model while penalizing model complexity \citep{JOHNSON2004101}. 

\textbf{I have just realized that this is an inappropriate use of AIC. These univariate models are not nested and are based on different datasets- different lagged meteorological data. I should actually use GCV- generalized cross validation}.

Once best-fit lags of meteorological variables for each dataset and temporal scale were determined, these lags were incorporated into multivariate models \ref{eq: multivariate1} and \ref{eq: multivariate2}. The generalized cross-validation (GCV) score and deviance explained of these multivariate models were compared. 

%GCV is a form of cross-validation procedure which quantifies the predictive capability of a model \citep{Chaves2019}. 


\pagebreak

\section{Results}

Figures/Charts: 

1. time series of abundance, temp, and precipitation
\begin{figure}[p]

		\begin{minipage}[]{.51\textwidth}
			\includegraphics[height = 3in, width=3.2in]{../Images/temp_ts.pdf}
			
			\hspace{4.3cm}\textbf{A.}\\
		\end{minipage}
		\begin{minipage}[]{.47\textwidth}
			\includegraphics[height = 3in, width=3.2in]{../Images/precip_ts.pdf}
			
			\hspace{4.4cm}\textbf{B.}\\
		\end{minipage}
	
	
		\begin{minipage}[]{.5\textwidth}
			\includegraphics[height = 3in, width=3.2in]{../Images/aedes_abun_ts.pdf}
			
			\hspace{4.3cm}\textbf{C.}\\
		\end{minipage}
		\begin{minipage}[]{.5\textwidth}
			\includegraphics[height = 3in, 			width=3.2in]{../Images/culex_abun_ts.pdf}
			
			\hspace{4.5cm}\textbf{D.}\\
		\end{minipage}
		
	\label{fig: ts_plots}
	\caption{Sample time series data of temperature, precipitation, and two vector species from Manatee County, Florida, 2012-2012. Data shown here are aggregated at the weekly scale. In \textbf{A.}, average daily maximum temperature per week shows regular seasonality with a range of about 15 \degree C. In \textbf{B.}, precipitation is very frequent during during a summer rainy season. This is common in all locations.\textbf{C.} and \textbf{D.} show patterns of abundance for two vector species, \textit{Aedes albopictus} and \textit{Culex nigripalpus}. Abundance datasets were non-continuous over winter periods where abundances are assumed to be low so traps are not employed.}

\end{figure}

2. partial dependency plot of multivariate GAM of a sample species 

\begin{figure}
	\begin{minipage}[]{0.5\textwidth}
		\includegraphics[height = 3.5in, width=3.5in]{../Images/multi_plotAQ.png}
		
		\hspace{4.5cm}\textbf{A.}\\
	\end{minipage}
	\begin{minipage}[]{0.5\textwidth}
	\includegraphics[height = 3.5in, width=3.5in]{../Images/multi_plotAA.png}
	
	\hspace{4.5cm}\textbf{B.}\\
	\end{minipage}

	\label{fig: 3Dmulti}
	\caption{This a sample of datasets fit with  multivariate models of temperature and precipitation at the best fit lags (Equation \ref{eq: multivariate1}). In \textbf{A.}, average \textit{Anopheles quadrimaculatus} count per trap at a weekly resolution was best fit according to univariate model comparison (Equations \ref{eq: univariateT} and \ref{eq: univariateP}) with temperature at a lag of six weeks and precipitation with a lag of four weeks. Both temperature and precipitation were significant. In \textbf{B.}, a Aedes albopictus abundance at a monthly resolution was best fit with temperature at a lag of one month  and precipitation in the contemporary month. In this dataset, only precipitation was significant.}

\end{figure}


\begin{figure}
	\centering
	\includegraphics[height = 3.5in, width=7in]{../Images/multi_plot2.png}
	\caption{This is my second option for demonstrating multivariate model fits (versus Fig. \ref{fig: 3Dmulti})}
\end{figure}

3. Table comparing deviance explained of different temporal resolutions

\begin{table}[h]
\begin{center}
\begin{tabularx}{.8\textwidth}{| L{0.5}  C{0.5}  C{.5}  C{0.5} | }
	\hline
	\multicolumn{4}{|l|}{Median Deviance Explained at Various Temporal Resolutions}\\
	\hline
	County & Weekly & Biweekly & Monthly \\
	\hline
	Lee & 31.9\% & 42.2\% & *52.2\% \\
	Manatee & 24.8\% & 21.6\% & *34.1\% \\
	Orange & 17.0\% & 15.8\% & *21.3\% \\
	Saint Johns & *8.5\% & 7.4\% & 8.4\% \\
	Walton & 23.3\% & 40.0\% & *49.9\% \\
	\hline
	All Counties & 20.0\% & 22.2\% & *31.9\% \\
	\hline
	
\end{tabularx}
\label{tab: bestres}
\caption{This table makes me hate Saint Johns}

\end{center}
\end{table}


5. Plot showing effect of autoregressive term. Also include in words what the AIC difference was between AR and non-AR model

\begin{figure}
	\centering
	\includegraphics[width= 5in, height = 3in]{../Images/devcomp.pdf}
\end{figure}
\pagebreak

\clearpage
\section{Discussion}

Mosquito-borne disease transmission is dependent on both natural and human environmental factors, such as climate, land use, health infrastructure, host demographics- using only climate will overestimate impact of climate change on populations \citep{Parham2015}

Aggregation of precipitation will lose the importance of quick torrential rains likely to flush immature mosquitoes \citep{Koenraadt2008}

Even if monthly aggregated data captures more of the variation, we're talking aggregation not surveillence protocol. Studies have shown lower sampling error when doing more frequent (1/week) sampling versus less frequent but more sampling (2/fortnight) \citep{MagbityLines2002}

%\end{linenumbers}

\bibliographystyle{apalike}
\bibliography{../../../Dropbox/BibFiles/Thesis.bib}

%TC:ignore
%\pagebreak
%\input{appendix.tex}
%TC:endignore

\end{document}