\documentclass[11pt, a4paper]{report}
\usepackage[margin = 2cm]{geometry}
\usepackage[utf8]{inputenc}
\usepackage{graphicx} %import graphics
% import pdf :\includegraphics[<options>]{filename.pdf}
\graphicspath{../../Results}
\usepackage{setspace}
\onehalfspacing % sets spacing to 1.5
\usepackage{titlesec}
\usepackage{caption}
\usepackage{subcaption}
\usepackage[export]{adjustbox} % adjust figure position
\usepackage{fixltx2e} % subscripts in text
\usepackage{natbib} %check out citation styles: https://www.economics.utoronto.ca/osborne/latex/BIBTEX.HTM
\setcounter{secnumdepth}{0} %removes numbering from all sections and levels below
\setlength{\parindent}{4em}
\usepackage{setspace}
\doublespacing
\usepackage{comment}
\usepackage[left]{lineno}
\usepackage{array}
\usepackage{gensymb} % For degree symbol
\usepackage{ltxtable}
\usepackage[table]{xcolor} 
\usepackage{amsmath, amssymb} % for expectation
\arrayrulecolor{gray}
\newcolumntype{L}[1]{>{\hsize=#1\hsize\raggedright\arraybackslash}X}
\newcolumntype{C}[1]{>{\hsize=#1\hsize\centering\arraybackslash}X}
\newcolumntype{R}[1]{>{\hsize=#1\hsize\raggedleft\arraybackslash}X}

% Change Figure to Fig.
\renewcommand{\figurename}{Fig.}

% Create expectation symbol
\DeclareMathOperator{\EX}{\mathbb{E}}

%Caption format
\captionsetup{
	width=\textwidth, % width of caption is 90% of current textwidth
	labelfont={bf, sc},        % the label, e.g. figure 12, is bold
	font=normalsize,          
	%format=hang,         % no caption text under the label
	justification=raggedright,
	singlelinecheck=false,
	labelsep=period
}


\titleformat{\section}{\normalfont\bfseries\LARGE}{}{0pt}{}
%\titleformat{\section}
%{\bfseries\Large} %formatting
%{} %numbering
%{0em} % distance between numbering and subtitle
%{} % anything to appear between # and title

\titleformat{\subsection}{\normalfont\bfseries\normalsize}{}{0pt}{}

\title{Evaluating Statistical Modelling Methodologies of Disease Vector Abundances } 

\author{Anne Marie Saunders\\
Imperial College London\\
MSc Computational Methods in Ecology and Evolution\\}

\date{August 27th, 2020\\ 
%Word Count: \input{sum.txt} words
}




\begin{document}
%\setcounter{page}{1}
%\pagenumbering{roman}

\begin{titlepage}
	\maketitle
\end{titlepage}


%\addcontentsline{toc}{section}{Declaration}
\renewcommand{\abstractname}{Declaration}
\begin{abstract}
Mosquito count data is from the VectDyn database of the VectorByte data platform. Meteorological data is from the NOAA Climate Data Online database. Meteorological data extraction from NetCDF files was adapted from protocols developed by Matthew Watts as part of the larger VectorByte project. All other data cleaning, processing, and analysis was my own work.
\end{abstract}

%\addcontentsline{toc}{section}{Abstract}
\renewcommand{\abstractname}{Abstract}
\begin{abstract}
Mosquito population dynamics directly affect disease incidence. Climate-based statistical models of abundance dynamics could provide early warnings of mosquito population growth and inform public health and mosquito management strategy. Existing attempts to model the relationship of temporally lagged meteorological variables on mosquito abundances tend to focus on single species in limited geographic areas and vary in their incorporation of auto régressive terms in the model.Conclusions about appropriate temporal  lag lengths and significance of meteorological predictors are consequently highly specific and limited in their potential to inform wider reaching early warning models. The temporal resolution of these models also range from weekly to monthly time scales with little consensus on appropriate time scale of analysis. In this study I used comprehensive multi-species surveillance data from 5 Florida counties to compare the performance of temperature and precipitation based GAMs of abundance when aggregated at weekly, bimonthly, and monthly time scales. I also investigated the consistency of best fit temporal lag lengths and significance of temperature and precipitation in predicting abundance across multiple locations for the same species. Lastly I assessed the relative effect on model fit of incorporating an auto-regressive term into climate-based GAMs. I found that in most locations, models parameterised with monthly aggregated data explain the most deviance in mosquito abundance dynamics with the least absolute error. Lags and significance of meteorological drivers of mosquito abundance were commonly shared within species, but varied enough to recommend that statistical models should be location-specific. According to AIC selection, autoregressive models were, in a majority of cases, better fit than non-autoregressive models, but a substantial 31.4\% of datasets were equally or better fit by climate-only models.
\end{abstract}

\clearpage\tableofcontents %creates page without page number
\thispagestyle{empty}

\pagebreak %also try \newpage

%\begin{linenumbers}
\section{Introduction}
\setcounter{page}{1}
%\pagenumbering{arabic}

%\subsubsection{Introduce}

At the end of the 20th century, the world experienced a global resurgence of vector-borne disease \citep{Gubler}. Diseases that had been well-controlled in the early-to-mid 1900s surged under complacent public health policies and insufficient research funding \citep{Gubler1998}. %Many of these diseases also emerged in novel geographic areas \citep{Gratz1999}. 
Today, vector-borne diseases represent 17\% of the total global infectious disease burden and cause millions of deaths annually. Mosquito-borne diseases, such as malaria, dengue, Zika, yellow fever, West Nile virus, and chikungunya, singularly infect more than an estimated 200 million individuals worldwide every year \citep{WHOreport}. 

Surveillance of mosquito populations is a successful method to control the public health impacts of vector-borne disease \citep{Vazquez-Prokopec2010}. Intervention in growing populations through chemical control measures can effectively reduce disease incidence \citep{Tomerini2011}. Sampling methods are, however, often limited by resource constraints \citep{Sedda2019}. Many studies have attempted to develop early warning models of disease incidence through prediction, rather than surveillance, of mosquito abundance \citep{Beck-Johnson2013, Li2019, Poh2019}. Because localised arbovirus disease data is often lacking in quality, mosquito abundance models can offer an alternative method to estimate disease risk \citep{Lowe2013}. Predictive models of mosquito abundance could also offer a cost-effective strategy with which to plan control measures \citep{Yang2009}.  
%In order to understand the potential applications of early warning mosquito abundance models, greater investigation is needed into optimal methodologies of vector abundance models across locations and species. 
%. 

Mosquito abundances are affected by many factors, including land-use, elevation, and vegetation cover, but meteorological variables, such as temperature and precipitation, are particularly predictive of population dynamics and commonly used in abundance models \citep{Trawinski2008, Li2019, Wang2011, Yoo2016}. The effects of temperature on mosquitoes are well characterised. As ectotherms, mosquito life history traits such as development rate, biting rate, fecundity, and survival, are temperature dependent and vary with changing temperatures \citep{Mordecai2019}. With increasing temperature, trait performance will increase until an optimal temperature is reached, after which, trait performance decreases according to physiological constraints \citep{Amarasekare2012}. Because these traits shape reproductive output, trait variation determines abundance of mosquito populations \citep{Cator2020}. Precipitation affects mosquitoes in more complex ways. Rainfall can create larval habitats in man-made containers or expand natural pooled breeding habitats \citep{Keith2005, Koenraadt2008}. Inter-seasonal variability in larval carrying capacity, which is dependent on rainfall for habitat creation, has been shown to be a main driver of mosquito abundances \citep{Marini2016}. Heavy rainfall, however, can flush immature mosquitoes from aquatic habitats, but the extent of this effect varies among species \citep{Koenraadt2008, Paaijmans2007}. The effect of droughts on abundance is mixed; in some cases, drought is thought to benefit mosquito abundance by eliminating predators of larvae in drying water bodies \citep{Chase2003}. Other research have found that droughts simply increase sample collection, rather than true abundance of mosquitoes \citep{Shaman2002}. In light of projected increases in extreme weather events such as increased heavy rainfall frequency, droughts, and warming temperatures in this century due to climate change, it is vital that we understand the complex effects of temperature and precipitation on vector populations. \citep{Seneviratne2012}.

A wide variety mechanistic and statistical modelling techniques have been used to investigate the impact of climate on mosquito abundance \citep{Ahumada2004, Cailly2012, Jian2014, Yoo2016, Wang2011}. Mechanistic models based on trait responses to climate conditions may allow for longer term and more generalisable forecasting \citep{Cator2020}. The complexity of the relationship between precipitation and mosquito life history, however, complicates the inclusion of this important abundance driver in mechanistic models. Statistical models, on the other hand, may be more useful for understanding  explanatory power of environmental variables \citep{Mordecai2019}. Mosquito abundances have been described phenomenologically by harmonic models, time series ARIMA models, Generalised Linear Models (GLMs), and Generalised Additive Models (GAMs) \citep{Li2019, Trawinski2008, Wang2011, Yoo2016}. GAMs have recently been used to capture the complex and non-linear relationships between climate and abundance as linear models are likely to overestimate the effect of temperature on mosquito abundance at high temperatures \citep{Li2019, Roiz2014, Xu2017}. 

Several unanswered methodological questions about the use of statistical models could limit the applicability of these methods as early warning systems of mosquito abundance and vector borne disease. Temperature and precipitation have both been found to have temporally lagged impacts on mosquito abundance at daily, weekly, and monthly scales \citep{Chuang2012, Poh2019, Xu2017, Wang2011}. In other words, past precipitation and temperature can drive contemporary mosquito population dynamics. Although lags of different time scales capture different effects of environmental drivers, tested methodologies for appropriate temporal resolutions to use for lagged models are non-existent \citep{Mordecai2019}. Most statistical models have also focused on characterising lags and dependence on meteorological variables in single species and limited geographic areas \citep{Chuang2012, Wang2011, Poh2019, Yoo2016}. Because abundance responses to meteorological drivers are also dependent on local land characteristics such as elevation, vegetation cover, and urban structures, climate effects are likely to be location-specific \citep{Ahumada2004, Yoo2016}. In order to understand the applicability of early warning systems for vector species abundances across geographies, the consistency of temporal lag lengths and meteorological abundance drivers in abundance models needs to be assessed across several locations. The inclusion of a first-order auto-regressive term (AR) of lagged abundance in analysis of GAMs can improve model fits by capturing density dependence of mosquito population dynamics and auto-correlative structure of mosquito abundances. \citep{DaCruzFerreira2017, Li2019, Xu2017}. The need for nearly contemporary abundance data for this AR term, however, could limit the real-time forecasting ability of these methods. Research is needed into the trade off in model fit with the use of AR terms to understand how applicable these models are in real time. This study aims to address these gaps in statistical modelling of mosquito abundance dynamics by answering the following questions:

\begin{enumerate}
	\item Which level of temporal resolution of temperature and precipitation data is best able to predict mosquito abundances?
	
	\item How consistently can temporal lags and significant meteorological drivers be characterised for species specific abundance models across multiple locations?
	
	\item How does incorporation of an auto-regressive term affect the predictive ability of temperature and precipitation driven models of mosquito abundance?
\end{enumerate}

To answer these questions, I will be evaluating temperature and precipitation dependent multivariate GAMs of mosquito abundance constructed using surveillance datasets from five geographically dispersed locations in Florida. This study is novel for its use of comprehensive multi-species and multi-location datasets. The insights gained from this investigation should help to inform the development of cost-effective early warning systems of increases of mosquito abundances and vector-borne-disease. 

%In this investigation I will take advantage of an open source mosquito abundance database to access comprehensive multi-species surveillance records in Florida, U.S.A. Florida's plentiful surveillance records reflect public health concerns stemming from a long history of mosquito-borne disease epidemics, including yellow fever, dengue, and malaria \citep{Connelly2014}. Since the start of the 21st century, this state has seen a re-emergence of dengue fever, geographically novel emergence of West Nile Virus, and locally transmitted outbreaks and incidences of Eastern Equine Encephalitis, St. Louis Encephalitis, Zika, and Chikungunya fever \citep{Connelly2014},\textbf{\cite{FloridaHealth2017}}. I first mapped openly-sourced datasets of precipitation and temperature data to Florida surveillance data and aggregated these datasets to weekly, bimonthly, and monthly time scales. I then fit univariate GAMs of the effect of each meteorological variable on mosquito abundance independently at multiple temporal lags. Model selection was used to find the best fit temporal lag length for each dataset, meteorological variable, and temporal resolution. Best fit lags of temperature and precipitation were then incorporated into a multivariate GAM of mosquito abundance. The fit and predictive ability of these multivariate models were compared across temporal resolutions. Consistency of  selected lag lengths and significance of meteorological drivers was then compared within 11 species across five Florida locations. Finally, I incorporated an AR term into multivariate GAMs and compared the relative fit and performance of auto-regressive and non-autoregressive models across datasets.



%More sustainable systems are needed for vector-borne disease surveillance \citep{Vazquez-Prokopec2010}

%At the end of the 20th century, the world experienced a rising global threat of vector-borne diseases (VBD) \citep{Gubler}. The incidence of many VBDs, such as malaria, yellow fever, and dengue, that had been locally eradicated in the early-to-mid 1900s, resurged under complacent public health policies and a lack of research funding \citep{Gubler1998} Of the many diseases threatening human health, mosquito borne diseases such as malaria, yellow fever, dengue, West Nile Virus, and Rift Valley fever take millions of lives every year \citep{Yang2009}. Mosquito abundances are affected by many factors, including land-use, elevation, and vegetation cover, but meteorological variables such as temperature and precipitation in particular can be predictors of population dynamics \citep{Yoo2016}. Rainfall produces basins of water for breeding while temperature mediates life-history processes at all life stages \citep{Yang2009, Beck-Johnson2013}. 
\pagebreak

\section{Methods}

\subsection{Mosquito Abundance Data}

Mosquito count data was obtained from the VectDyn database (\textbf{HOW TO CITE?}). VectDyn is a global database containing spatially and temporally explicit abundance data of mosquitoes and other arthropod vectors. Mosquito abundance from 205 global locations was narrowed down to 7 data-rich counties in Florida, U.S.A. From these 7 counties, 5 counties were determined to have nearly year-round sampling from which fairly continuous time series of mosquito abundances could be formed. 

\begin{center}
	\begin{tabularx}{4in}{ | L{1.4} | C{0.6} | }
		\hline
		\textbf{County} & \textbf{Years of Data} \\\hline
		Lee & 11 \\\hline
		Manatee & 5 \\\hline
		Orange & 6 \\\hline
		Saint Johns & 13.5 \\\hline
		Walton & 3 \\\hline 
		\multicolumn{1}{|r|}{\textbf{Average}} & \textbf{7.7} \\\hline
	\end{tabularx}\\
	\captionof{table}{\label{Tab:Locs} Locations of abundance data and associated length of data location at each location. Individual species in these locations may have abundance records that are shorter than the overall collection duration of each location. }
\end{center}	

Each location contained multiple trap sites. A variety of trap types were used, including BG-Sentinel traps, CDC light traps, animal-baited traps, CDC gravid traps, and \textbf{collection of arthropods}. Because I was only concerned with relative changes in abundance and not absolute abundance across species or locations, I included all trap types in my data. Abundance was recorded in integer count values and usually identified to the species level. 


\subsection{Meteorological Data}

Temperature and precipitation datasets were obtained from the NOAA Climate Data Online database as NetCDF raster files at a spatial resolution of 0.50 degrees latitude and 0.50 degrees longitude. Maximum temperature in Celsius and total daily precipitation in millimetres were used based on availability of data. Rasters were rotated 180 degrees to match coordinate rotation of trap locations. Maximum daily temperature and total precipitation values were then extracted by taking the mean of the bilinear interpolation of the 4 closest raster cells to each trap location. I then mapped extracted maximum temperature and total precipitation values to corresponding mosquito abundances by date and trap location. 

\textbf{Number of days of rainfall has been shown to be a more effective predictor of mosquito-borne disease incidence than cumulative precipitation \citep{Xu2017}. This is likely due to the maintenance of humid conditions over time with frequent rainfall. Humidity has been independently assessed as a significant predictor of abundance dynamics \citep{Trawinski2008}, and so this representation of precipitation may capture both humidity and precipitation effects.}

\subsection{Data Pre-Processing}

In order to account for discrepancies between true zero count instances and NA values in abundance data, I set abundance for each species to zero where at least one mosquito of any species had been caught at the same trap on the same day. 

I then spatially and temporally aggregated meteorological data and species-level abundance data. At the spatial level, I averaged the maximum temperature, total precipitation, and species-specific abundance from trap-specific to county-wide. This transformed integer count values to averaged indicators of overall abundance for the county-level spatial scale. I aggregated morphological groups of non-differentiable species that could be easily mis-identified (\textbf{list in SI}). I then removed species that had only zero or NA abundance values. Species counts that were identified only to the genus or family level were also removed. I then temporally aggregated maximum temperature, total precipitation, and species-specific abundance by averaging at weekly, biweekly, and monthly scales. Consequently, temperature refers to the average maximum daily temperature across respective temporal scales and precipitation refers to average daily precipitation across the respective temporal scales. Finally, I removed rows of data with missing values in response or explanatory variables. 

\textbf{include paragraph on preprocessing for GAM: dealing with NAs, interpolation for autoregressive model}

\subsection{Model Structure}

I used univariate generalized additive models for each species at each location to determine the best-fit temporal lags between abundance and maximum temperature as well as between abundance and precipitation. Because aggregated abundance values are positive, non-integer, and non-normally distributed, I used a Gamma family distribution with a log-link function. These models had the form:
\begin{equation}
	ln(V_t + 1) = a_0 + f_1(T_{t-l_T}) + \epsilon_t
	\label{eq: univariateT}
\end{equation}
\begin{equation}
	ln(V_t + 1) = a_0 + f_1(P_{t-l_P}) + \epsilon_t
	\label{eq: univariateP}
\end{equation}


$V_t$ is abundance at time $t$, $a_0$ is the intercept, $T_{t-l_T}$ and $P_{t-l_P}$ are the temperature and precipitation at $l_T$ and $l_P$, respectively, time periods prior to time $t$. One is added to abundance at time $t$ in order to prevent undefined values from the logarithm of zero abundance values. $f_1$ is a smooth function comprised of cubic polynomial basis functions. I set the number of basis functions for each smooth function to four throughout this analysis in order to control overfitting of the GAM. 

Each dataset was also fit with two multivariate models, each incorporating temperature and precipitation, and one of which incorporating an autoregressive term. 

\begin{equation}
	ln(V_t + 1) = a_0 + f_1(T_{t-l_T}) + f_2(P_{t-l_P}) + \epsilon_t
	\label{eq: multivariate1}
\end{equation}
\begin{equation}
	ln(V_t + 1) = a_0 + f_1(T_{t-l_T}) + f_2(P_{t-l_P}) + f_3(V_{t-1} + 1) + \epsilon_t
	\label{eq: multivariate2}
\end{equation}

In Equation \ref{eq: multivariate2}, $V_{t-1}$ is the abundance one time step prior to abundance at time $t$. 

\subsection{Model Selection and Evaluation}

Models \ref{eq: univariateT} and \ref{eq: univariateP} were used to determine the best-fit lags of temperature and precipitation for each dataset at each temporal aggregation. At the weekly and biweekly scale, lags between zero and six weeks were considered. At the monthly scale, lags between zero and two months were considered. Best-fit lags for each temporal scale were chosen by finding the minimum Akaike's Information Criterion (AIC) value among models fit with each possible lag length. 

\begin{equation}
	AIC = -2ln[L(\hat{\theta_{p}}|y)] + 2p
	\label{eq: AIC}
\end{equation}

AIC is a method for determining the likelihood of a given model while penalizing model complexity \citep{JOHNSON2004101}. 

\textbf{I have just realized that this is an inappropriate use of AIC. These univariate models are not nested and are based on different datasets- different lagged meteorological data. I should actually use GCV- generalized cross validation}.

Once best-fit lags of meteorological variables for each dataset and temporal scale were determined, these lags were incorporated into multivariate models \ref{eq: multivariate1} and \ref{eq: multivariate2}. The generalized cross-validation (GCV) score and deviance explained of these multivariate models were compared. 

%GCV is a form of cross-validation procedure which quantifies the predictive capability of a model \citep{Chaves2019}. 


\pagebreak

\section{Results}

Each temporal resolution contained 161 datasets of unique species and location combinations in each temporal resolution. 50 weekly, 42 bimonthly, and 32 monthly datasets were removed that where less than 10\% of the data was non-zero. Time series of temperature, precipitation, and abundance data for two sample datasets can be found in Fig \ref{fig: ts_plots}. 

%\subsection{Lag Selection}
%Selection by AIC of the best fit temperature lags in univariate models of mosquito abundance  (Equation \ref{eq: univariateT}) revealed that temperature in the contemporary time period ($l_T = 0$) was most frequently the best fit lag across all temporal resolutions. For precipitation models (Equation \ref{eq: univariateP}), lags of one and two weeks at the weekly scale were nearly equally most frequent (selected in 14 and 13 datasets, respectively). At the bimonthly scale, best fit lags of zero lags were most frequent but were favoured in only one more dataset than precipitation with one half month lag (31 and 30 datasets, respectively). At the monthly scale, the contemporary month was most frequently the best fit lag.

% Multivariable partial dependencies
\begin{figure}[h!]
	\begin{minipage}[]{\textwidth}
		\centering
		\includegraphics[height = 3in, width=6in]{../Images/multi_plotAQ.pdf}
	\end{minipage}
	\begin{minipage}[]{\textwidth}
		\vspace{.3cm}
		\hspace{5.2cm}\textbf{A.}
		\hspace{7cm}\textbf{B.}
	\end{minipage}
	\caption{Sample of multivariate model fitting  showing the partial dependency of \textit{Anopheles quadrimaculatus} abundance in Lee County, Florida on temperature and precipitation at the monthly temporal aggregation. Both temperature and precipitation are significant and best fit at a lag of two months.}
	\label{fig: multivar}
\end{figure}

% Time series plots
\begin{figure}[h!]
	\begin{minipage}[]{.48\textwidth}
		\includegraphics[height = 3in, width=3.2in]{../Images/temp_ts.pdf}
		
		\hspace{4.3cm}\textbf{A.}\\
	\end{minipage}
	\begin{minipage}[]{.52\textwidth}
		\includegraphics[height = 3in, width=3.4in]{../Images/precip_ts.pdf}
		
		\hspace{4.4cm}\textbf{B.}\\
	\end{minipage}
	
	
	\begin{minipage}[]{.5\textwidth}
		\includegraphics[height = 3in, width=3.2in]{../Images/aedes_abun_ts.pdf}
		
		\hspace{4.3cm}\textbf{C.}\\
	\end{minipage}
	\begin{minipage}[]{.5\textwidth}
		\includegraphics[height = 3in, 			width=3.2in]{../Images/culex_abun_ts.pdf}
		
		\hspace{4.5cm}\textbf{D.}\\
	\end{minipage}
	
	\caption{Time series data of temperature, precipitation, and two vector species from Manatee County, Florida, 2012-2016. Data shown here are aggregated at the monthly scale. In \textbf{A.}, average daily maximum temperature per week shows regular seasonality with a range of about 15 \degree C. In \textbf{B.}, precipitation is very frequent during during a summer rainy season. This is common in all locations. \textbf{C.} and \textbf{D.} show patterns of abundance for two vector species, \textit{Aedes albopictus} and \textit{Culex nigripalpus}. Abundance datasets were non-continuous over winter periods where abundances are assumed to be low so traps are not employed. These missing data points were removed from datasets prior to analysis.}
	\label{fig: ts_plots}
\end{figure}

\subsection{Which level of temporal resolution of temperature and precipitation data is best able to predict mosquito abundances?}

The fit of multivariate models (Fig \ref{fig: multivar}) of temperature and abundance at best fit lags were compared across weekly, bimonthly, and monthly resolutions (Table \ref{tab: bestresdevexplained}) using MAE and deviance explained. 111 out of the original 161 datasets were used for this comparison. The fifty datasets that were removed had fewer than 10\% non-zero data points or did not converge to a multivariate model due to a lack of sufficient data in at least one of the temporal resolutions. In all locations besides St. Johns, monthly aggregated datasets had a higher median deviance explained than the median deviance explained at weekly and bimonthly resolutions. Three out of five locations had the lowest median MAE at the monthly resolution, indicating that monthly datasets had tended to have less prediction error than other resolutions. St Johns datasets had equal median MAE at every resolution, while Lee datasets had the lowest median error at the weekly resolution. Because monthly datasets could explain the most deviance in four out of five locations, and had the lowest median MAE in three out of five locations, this resolution was used for the rest of the analysis.

% MAE table
\begin{table}[h!]
	\begin{center}
		\begin{tabularx}{.9\textwidth}{| L{0.5}  C{0.5}  C{.5}  C{0.5} | }
			\hline
			\multicolumn{4}{|l|}{Median Deviance Explained and Median MAE at Various Temporal Resolutions}\\
			\hline
			County & Weekly & Bimonthly & Monthly \\
			\hline
			Lee & 37.7\% (3.60) & 45.8\% (9.66)& 52.2\% (11.87)\\
			Manatee & 30.2\% (1.04)& 35.5\% (0.92)& 40.1\% (0.90)\\
			Orange & 26.2\% (0.50)& 29.0\% (0.44)& 35.0\% (0.43)\\
			St. Johns & 13.0\% (0.20)& *25.9\% (0.20)& 12.6\% (0.20)\\
			Walton & 31.7\% (0.26)& 40.2\% (0.22)& 49.9\% (0.19)\\
			\hline
			All Counties & 28.6\% (0.76)& 33.0\% (0.83)& 37.9\% (0.78)\\
			\hline
			%\multicolumn{4}{l}{\small *marks the best-performing temporal scale in each row} \\
			
		\end{tabularx}
		\caption{Deviance explained and MAE by the best fit multivariate model of temperature and precipitation for each dataset summarised by the median value at each location and temporal resolution. MAE is in parentheses. Median was used because the distribution of deviance explained and MAE was left skewed.}
		\label{tab: bestresdevexplained}
	\end{center}
\end{table}

\subsection{How consistent are best fit temporal lags and the influence of temperature and precipitation on mosquito abundance across locations?}

I used corrected AIC selection in univariate models of temperature and precipitation to find the best fit lags of temperature and precipitation for each dataset. Temperature in the contemporary time period ($l_T = 0$) was most frequently the best fit lag across all temporal resolutions. For precipitation models (Equation \ref{eq: univariateP}), lags of one and two weeks at the weekly scale were nearly equally most frequent (selected in 14 and 13 datasets, respectively). At the bimonthly scale, best fit lags of zero lags were most frequent but were favoured in only one more dataset than precipitation with one half month lag (31 and 30 datasets, respectively). At the monthly scale, the contemporary month was most frequently the best fit lag.

In order to assess the consistency of best-fit meteorological model characteristics in individual species across locations, I narrowed my monthly dataset to eleven species that occurred in all five locations and examined the frequency of best-fit lags and significant variables in each species. Nine out of eleven species (81.8\%) had single temperature lag that was preferred in a majority of locations (Fig \ref{fig: acrosslocation}A.). Only one species, \textit{Psorophora ciliata}, had consistent best fit lags for temperature across all locations. Eight of the eleven species had a precipitation lag that was best fit for a majority of the locations, but only 1 species, \textit{Aedes infirmatus}, had a single lag length consistently chosen across all locations (Fig \ref{fig: acrosslocation}B.).

Across 128 monthly datasets with sufficient non-zero values and convergence in a multivariate model of abundance, models with significance in both temperature and precipitation was slightly more common than other combinations of variables. Variable categories were sorted by number of datasets where only temperature (25.8\%), only precipitation (25.0\%), both variables (30.5\%), or neither variable (18.8\%) was significant (Fig \ref{fig: acrosslocation}C.). In the focal eleven species, there were no species with a single category of significant predictors consistent across locations. Eight out of eleven species (72.7\%), however, had a single category of significant predictors in at least a majority of locations.  



% Old Significance figure with genus
\begin{comment}

\begin{figure}
	
	\begin{minipage}{\textwidth}
		\hspace{.1\textwidth}
		\textbf{\Large  \textcolor{white}{A.} \hspace{.5\textwidth}B.}
	\end{minipage}
	\begin{minipage}{.5\textwidth}
		\begin{center}
			\includegraphics[width=1.5in, height=1.5in]{../Images/legend_sig.pdf}
		\end{center}
		\vspace{1cm}
		\hspace{.2\textwidth}\textbf{\Large A.}
		
		\includegraphics[width=3in, height=1.5in]{../Images/alldatasets_sig.pdf}
	\end{minipage}
	\begin{minipage}{.5\textwidth}
		\includegraphics[width=3.5in, height=4in]{../Images/bygenus_sig.pdf}
	\end{minipage}
	
	\begin{minipage}{\textwidth}
		\hspace{.1\textwidth}
		\textbf{\Large  C.\hspace{.5\textwidth}D.}
	\end{minipage}
	\begin{minipage}{.5\textwidth}
		\includegraphics[height=4in, width=3.4in]{../Images/byspecies4_sig.pdf}
	\end{minipage}
	\begin{minipage}{.5\textwidth}
		\includegraphics[height=4in, width=3.4in]{../Images/byspecies5_sig.pdf}
	\end{minipage}
	\caption{Significance of meteorological variables in multivariate GAMs of mosquito abundance. "Precipitation" describes datasets with only precipitation as a significant predictor while "Temperature" describes datasets with only temperature as a significant predictor. \textbf{A.} shows the significances of each variable across 128 monthly datasets. In \textbf{B.}, species are sorted by genus and significance of meteorological variables is shown. Genus groupings often contain multiple datasets of the same species at different locations. In \textbf{C.} and \textbf{D.}, species-level significance are shown in 23 species that were present in four or five, respectively, of the five studied locations.}
	\label{fig: significance}
\end{figure}
\end{comment}


% Table I worked so hard for
\begin{comment}

\begin{table}[h!]
\begin{center}
\caption{Significance of temperature and precipitation in seven major vector species across Florida locations according to a multivariate GAM  constructed with the best fit lags of each variable at a monthly resolution. Each species was present in every location. Significance is marked with by the presence of a T for temperature and P for precipitation. Total locations with significant temperature and precipitation are summed for each species}
\begin{tabularx}{\textwidth}{| L{0.25} | C{0.25}  C{.25}  C{0.25} C{0.25} C{.25}  C{0.25} | }
\hline
%\multicolumn{7}{|l|}{Significance of Temperature and Precipitation Across Locations}\\
%\hline
\textbf{Locations}& \small{\textit{Aedes\textsuperscript{1}}} & \textit{\small Aedes\textsuperscript{2}} & %\textit{\small Aedes\textsuperscript{3}} &
\textit{\small Anopheles\textsuperscript{1}} & \textit{\small Culex\textsuperscript{1}} & \textit{\small Culex\textsuperscript{2}} & \textit{\small Culiseta\textsuperscript{1}} \\\hline

%\textbf{Locations} &&&&&&& \\
Lee & \textcolor{white}{T}\quad P & \textcolor{white}{T}\quad P %& T\quad P 
& T\quad P & T\quad \textcolor{white}{P} & T\quad P & T\quad P \\

Manatee & T\quad \textcolor{white}{P} & T\quad P & T\quad P %& T\quad \textcolor{white}{P} 
& T\quad \textcolor{white}{P} &  T\quad P &T\quad \textcolor{white}{P} \\

Orange & T\quad P & T\quad \textcolor{white}{P} 
%& T\quad \textcolor{white}{P} 
& T\quad \textcolor{white}{P} & T\quad \textcolor{white}{P} &  T\quad P &\textcolor{white}{T}\quad P \\

St. Johns & T\quad P & \textcolor{white}{T}\quad P %& T\quad P 
&  & T\quad P 
&  \textcolor{white}{T}\quad P &\textcolor{white}{T}\quad P \\

Walton & T\quad P & T\quad P &% & 
& T\quad \textcolor{white}{P} & T\quad \textcolor{white}{P} & T\quad P \\\hline

\textbf{Total P} & \textcolor{white}{T}\quad 4 &  \textcolor{white}{T}\quad 4 & %\textcolor{white}{T}\quad 3 &
\textcolor{white}{T}\quad 1 & \textcolor{white}{T}\quad 1 &
\textcolor{white}{T}\quad 4 &
\textcolor{white}{T}\quad 4 \\\hline

\textbf{Total T} & 4
\quad \textcolor{white}{P} & 3 \quad \textcolor{white}{P} %& 4 \quad \textcolor{white}{P} 
& 3 \quad \textcolor{white}{P} & 5 \quad \textcolor{white}{P} & 4 \quad \textcolor{white}{P} & 3 \quad \textcolor{white}{P} \\\hline

\multicolumn{7}{l}{}\\

\multicolumn{7}{l}{\textit{Aedes\textsuperscript{1} = Aedes atlanticus tormentor morphological group; Aedes\textsuperscript{2} = Aedes albopictus;}}\\

\multicolumn{7}{l}{\textit{Anopheles\textsuperscript{1} = Anopheles quadrimaculatus; Culex\textsuperscript{1} = Culex pipiens morphological group;}}\\

\multicolumn{7}{l}{\textit{Culex\textsuperscript{2} = Culex nigripalpus; Culiseta\textsuperscript{1} = Culiseta melanura}} \\

\end{tabularx}

\label{tab: compare_sites}
\end{center}
\end{table}
\end{comment}

\subsection{How does incorporation of an autoregressive term affect the predictive ability of temperature and precipitation driven models of mosquito abundance?}

I used relative comparison of deviance explained and corrected AIC to understand the change in multivariate model fits with the addition of an autoregressive term of lagged abundance (Equation \ref{eq: multivariate2}). From 129 monthly datasets, 8 were removed that failed to converge in either multivariate model. As expected, the autoregressive term tended to improve model fit with a mean change in deviance explained of +16.7\% and median change in deviance explained of +10.9\% across all locations. Four datasets had a decrease in deviance explained with incorporation of an autoregressive term (range of [-2.1\%, -0.2\%]). Surprisingly, according to selection through corrected AIC, almost a third of datasets (31.4\%) were equally or better fit by the model with only meteorological predictors.

% AR versus non AR
\begin{figure}[h!]
	\begin{minipage}{.5\textwidth}
		\centering
		\includegraphics[width=3.2in, height=3in]{../Images/devcomp.pdf}
	\end{minipage}
	\begin{minipage}{.5\textwidth}
		\centering
		\includegraphics[width=3in, height=3in]{../Images/AICcomp.pdf}
	\end{minipage}
	\begin{minipage}{\textwidth}
		\vspace{.25cm}
		\hspace{.25\textwidth}
		\textbf{A.}
		\hspace{.47\textwidth}
		\textbf{B.}
	\end{minipage}
	\caption{Comparison of the fit of autoregressive versus non-autoregressive meteorological models of mosquito abundance for each monthly dataset. After removing 8 datasets where the autoregressive model did not converge, 121 datasets were compared. \textbf{A.} is a histogram of the relative change in deviance explained with the incorporation of an autoregressive term to the multivariate model for each dataset. Four datasets had a decrease (range of [-2.1\%, -0.2\%]) in deviance explained with incorporation of an autoregressive term. In \textbf{B.}, the number of datasets where the autoregressive model was a better, equal, and worse fit than the non-autoregressive model is shown. 
	}
\end{figure}

% New significance figure with lags
\begin{figure}
	\begin{minipage}{\textwidth}
		\hspace{.1\textwidth}
		\textbf{\Large  A.\hspace{.5\textwidth}B.}
	\end{minipage}
	\begin{minipage}{.65\textwidth }%4.3in}
		\includegraphics[height=4in, width=4.3in, left ]{../Images/byspecies5_templags.pdf}
	\end{minipage}
	\begin{minipage}{2.7in}
		\includegraphics[height=4in, width=2.7in]{../Images/byspecies5_preciplags.pdf}
	\end{minipage}
	\vspace{1cm}
	
	\begin{minipage}{\textwidth}
		\hspace{.1\textwidth}
		\textbf{\Large  \hspace{.47\textwidth}D.}
	\end{minipage}
	\begin{minipage}{.3\textwidth}
		%\centering
		\includegraphics[width=1.5in, height=1.25in]{../Images/legend_sig.pdf}
		
		%\vspace{.4cm}
		%\hspace{.1\textwidth}
		\textbf{ \Large C.}
		\includegraphics[width=2in, height=2.5in]{../Images/alldatasets_sig.pdf}
	\end{minipage}
	\begin{minipage}{.7\textwidth}
		\includegraphics[height=4in, width=5in]{../Images/byspecies5_sig.pdf}
	\end{minipage}
	
	\caption{Frequency of best fit lags and significant variables across locations for monthly datasets. Morphological groups are marked with asterisks. In \textbf{A.} and \textbf{B.}, frequency of best fit temperature and precipitation lags chosen by corrected AIC from univariate models of abundance are shown for eleven species that were present in all locations. \textbf{C.} shows the frequency of significant predictors for 128 abundance datasets fit with multivariate models of temperature and precipitation. In \textbf{D.}, the eleven species are shown with the frequency of significant variables in datasets across all locations.}
	\label{fig: acrosslocation}
	
\end{figure}

\pagebreak

\clearpage
\section{Discussion}

Mosquito-borne disease transmission is dependent on both natural and human environmental factors, such as climate, land use, health infrastructure, host demographics- using only climate will overestimate impact of climate change on populations \citep{Parham2015}

Aggregation of precipitation will lose the importance of quick torrential rains likely to flush immature mosquitoes \citep{Koenraadt2008}

Even if monthly aggregated data captures more of the variation, we're talking aggregation not surveillence protocol. Studies have shown lower sampling error when doing more frequent (1/week) sampling versus less frequent but more sampling (2/fortnight) \citep{MagbityLines2002}

%\end{linenumbers}

\bibliographystyle{apalike}
\bibliography{../../../Dropbox/BibFiles/Thesis.bib}

%TC:ignore
%\pagebreak
%\input{appendix.tex}
%TC:endignore

\end{document}