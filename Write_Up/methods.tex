\section{Methods}

\subsection{Mosquito Abundance Data}

Mosquito count data was obtained from the VectDyn database (\textbf{FIXME CITE}). VectDyn is a global database containing spatially and temporally explicit abundance data of mosquitoes and other arthropod vectors from published data and surveillance program records. Mosquito abundance from 205 global locations was narrowed down to seven data-rich counties in Florida, U.S.A. I decided to use data from five of these counties which had multi-year surveillance records and nearly year-round sampling from which fairly continuous time series of mosquito abundances could be formed. 

Each location contained multiple trap sites. A variety of trap types were used, including BG-Sentinel traps, CDC light traps, animal-baited traps, and CDC gravid traps. Trap type can affect the efficacy of trapping for different mosquito species \citep{Li2016}. At locations with multiple traps, proportionality of each trap type was inconsistent across the time series. This would complicate conclusions of species abundances aggregated across multiple trap types. Thus, for locations with multiple types of traps, I identified the most common trap type and removed observations from all other trap types. 

%In order to account for discrepancies in data entry between true zero count instances and NA values in abundance data, I set abundance for each species to zero where at least one mosquito of any species had been caught at the same trap on the same day. 

I averaged species-specific count data in each location and at weekly, biweekly, and monthly temporal resolutions. This transformed integer count values to averaged indicators of county-level abundance and allowed me to account for frequent variation in the number of traps deployed for each location. I aggregated species that are morphologically indistinguishable from one another according to the \textbf{ontology used by VectorBase- FIXME CITE/REFERENCE} (\textbf{list in SI}). I then removed species that had only zero or NA abundance values. Species counts that were identified only to the genus or family level were also removed. After this processing, I had 163 datasets of unique species and location combinations aggregated at weekly, biweekly, and monthly resolutions. Table \ref{Tab:Locs} summarises the results of this processing at my five focal locations.

\begin{center}
	\begin{tabularx}{7in}{ | L{0.4} | L{0.4} | L{0.4} | L{0.4} | L{0.4} |}
		\hline
		\textbf{County} & \textbf{Trap Type} &\textbf{Years of Data} & \textbf{Observations} & \textbf{Number of Species}\\\hline
		Lee & CDC Light Trap &11 & & 18\\\hline
		Manatee & CDC Light Trap & 5 & & 48\\\hline
		Orange & CDC Light Trap & 6 & & 31\\\hline
		Saint Johns & baited light trap & 13.5 & & 38\\\hline
		Walton & New Jersey Trap &3 & & 28\\\hline 
	\end{tabularx}\\
	\captionof{table}{\label{Tab:Locs} Locations of abundance data and associated length of data collection, trap type, and number of species-level observations at each location. Individual species in these locations may have abundance records that are shorter than the overall years of data of each location. \textbf{FIXME table formatting}}
\end{center}	

\subsection{Meteorological Data}

Temperature and precipitation datasets were obtained from the NOAA Climate Data Online database as global NetCDF raster files at a spatial resolution of 0.50 degrees latitude and 0.50 degrees longitude. Maximum temperature in Celsius and total daily precipitation in millimetres were used based on availability of data. Rasters were rotated 180 degrees to match coordinate rotation of trap locations. Maximum daily temperature and total precipitation values were then extracted by taking the mean of the bilinear interpolation of the 4 closest raster cells to each trap location. Each raster file contained one year of meteorological data, and so this procedure was repeated for each year of the surveillance period. I then mapped the extracted daily maximum temperature and total daily precipitation to corresponding mosquito abundances by date and trap location. 


\subsection{Data Pre-Processing}

I spatially and temporally aggregated maximum temperature data by averaging both at county-level and at weekly, biweekly, and monthly time scales. Consequently, temperature in this study refers to the average maximum daily temperature across respective temporal scales.  \textbf{FIXME explain why choice of max temp vs average, min, etc}

Number of days of rainfall has been shown to be a more effective predictor of mosquito-borne disease incidence than cumulative precipitation \citep{Xu2017}. This is likely due to the maintenance of humid conditions over time with frequent rainfall. Humidity has been independently assessed as a significant predictor of abundance dynamics \citep{Trawinski2008}, and so this representation of precipitation may capture both humidity and precipitation effects. With this in mind, I aggregated precipitation to weekly, biweekly, and monthly scales by summing the number of days in each temporal period with non-zero cumulative precipitation. Consequently precipitation as I will further refer to it refers to a discrete number of days of rainfall, with a maximum of 7, 14, or 31 dependent on temporal scale. Finally, I removed rows of data with missing values in any response or explanatory variables. 


\subsection{Model Structure}

I used univariate generalized additive models for each dataset to determine the best-fit temporal lags between abundance and temperature as well as between abundance and precipitation. Because aggregated abundance values are positive, non-integer, and non-normally distributed, I used a Gamma family distribution with a log-link function. These models had the form:
\begin{equation}
	ln(V_t + 1) = a_0 + f_1(T_{t-l_T}) + \epsilon_t
	\label{eq: univariateT}
\end{equation}
\begin{equation}
	ln(V_t + 1) = a_0 + f_1(P_{t-l_P}) + \epsilon_t
	\label{eq: univariateP}
\end{equation}


$V_t$ is abundance at time $t$, $a_0$ is the intercept, $T_{t-l_T}$ and $P_{t-l_P}$ are the temperature and precipitation at $l_T$ and $l_P$, respectively, time periods prior to time $t$. One is added to abundance at time $t$ in order to prevent undefined values from the logarithm of zero abundance values. $f_1$ is a smooth function comprised of cubic polynomial basis functions. The use of cubic polynomial basis functions has been shown as an effective way to avoid the bias introduced by concurvity with non-parametric basis functions\citep{Dominici2002, Ramsay2003}. Concurvity is the extent to which each predictive smooth function can be approximated by other smooth function predictors and is analogous to multicollinearity in linear models. This can cause unstable estimates of the response variable and cause underestimation of standard error of explanatory variables \citep{Ramsay2003}. 
%I restricted the number of basis functions to nine for most smooth functions. This allowed me to generalise control for overfitting of the GAM across all datasets. Precipitation smooth functions for datasets with fewer than nine unique values for precipitation were set to the maximum number of basis functions allowable. This is equal to the number of unique values of the variable. This was mostly applicable to weekly datasets, which could have a maximum of eight unique values for precipitation.

Each dataset was also fit with two multivariate models, each incorporating temperature and precipitation, and one of which incorporating an autoregressive term.

\begin{equation}
	ln(V_t + 1) = a_0 + f_1(T_{t-l_T}) + f_2(P_{t-l_P}) + \epsilon_t
	\label{eq: multivariate1}
\end{equation}
\begin{equation}
	ln(V_t + 1) = a_0 + f_1(T_{t-l_T}) + f_2(P_{t-l_P}) + f_3(V_{t-1} + 1) + \epsilon_t
	\label{eq: multivariate2}
\end{equation}

In Equation \ref{eq: multivariate2}, $V_{t-1}$ is the abundance one time step prior to abundance at time $t$. 


\subsection{Model Selection and Evaluation}

Models \ref{eq: univariateT} and \ref{eq: univariateP} were used to determine the best-fit lags of temperature and precipitation for each dataset at each temporal aggregation. At the weekly and biweekly scale, lags between zero and six weeks were considered. At the monthly scale, lags between zero and two months were considered. Best-fit lags for each temporal scale were chosen by finding the minimum Akaike's Information Criterion (AIC) value among models fit with each possible lag length. 

\begin{equation}
	AIC = -2ln[L(\hat{\theta_{p}}|y)] + 2p
	\label{eq: AIC}
\end{equation}

AIC is a method for determining the likelihood of a given model while penalizing model complexity \citep{JOHNSON2004101}. 

The Akaike weight of each best fit model was calculated in order to indicate the confidence in this lag as compared to other univariate lagged models. 

\textbf{FIXME- I have just realized that this is an inappropriate use of AIC. These univariate models are not nested and are based on different datasets- different lagged meteorological data. I should actually use GCV- generalized cross validation}.

Once best-fit lags of meteorological variables for each dataset and temporal scale were determined, these lags were incorporated into multivariate models \ref{eq: multivariate1} and \ref{eq: multivariate2}. The generalized cross-validation (GCV) score and deviance explained of these multivariate models were compared. 

%GCV is a form of cross-validation procedure which quantifies the predictive capability of a model \citep{Chaves2019}. 

